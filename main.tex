\documentclass [11pt]{article}
\renewcommand{\baselinestretch}{1.1}

%%% Add packages here
\usepackage{graphicx} 
\usepackage{parskip}
\usepackage[table]{xcolor}
\usepackage{amsmath}
\usepackage{float}
\usepackage{geometry}
    \geometry{legalpaper, portrait, margin=3cm}
\usepackage{makecell}
\usepackage{tabularx}
\usepackage{multirow}
\usepackage{array}
\usepackage{amsmath}
\usepackage{makecell}

	\usepackage{lscape}
	\usepackage{amsfonts}
	\usepackage{amsmath}
	\usepackage{amsthm}
	\usepackage{amssymb}
	\usepackage{latexsym}
	\usepackage{color}
	\usepackage{verbatim}
	\usepackage{fancyhdr}
	\usepackage{fancybox}
    % \usepackage{nameref}
        \usepackage{listings} % for R code setup
        %\usepackage{xcolor}
        \usepackage{threeparttable} % to use table notes
        \usepackage{hologo} % for TeX engine logos
        \usepackage{booktabs} % for nice tables
        \usepackage{float} % for floating table
        \usepackage{array} % centerly aligned table column
        \usepackage{cite} % for citation
        \usepackage{longtable} % for the 3rd table
        \usepackage[table,xcdraw]{xcolor}  % for the 3rd table
        %\usepackage{titlesec}
        \usepackage{appendix} % for adding appendix

%%%%%%%%%%%%%%%%%%%%%%%%%%%%%%%%%%%%%

% Reduce space before and after sections
%\titlespacing{\section}{0pt}{*1}{*0.5}
%\titlespacing{\subsection}{0pt}{*0.5}{*0.3}
%\titlespacing{\subsubsection}{0pt}{*0.3}{*0.2}

%%% Margins
\addtolength{\oddsidemargin}{-.50in}
\addtolength{\evensidemargin}{-.50in}
\addtolength{\textwidth}{1.0in}
\addtolength{\topmargin}{-.40in}
\addtolength{\textheight}{0.80in}

%%% Header
	\pagestyle{fancy}
	\chead{\groupname}
	\rhead{}
	\lhead{}
	\cfoot{\thepage}
	\renewcommand{\headrulewidth}{0.4pt}
%%%%%%%%%%%%%%%%%%%%%%%%%%%%%%%%%%%%%

\begin{document}
\clearpage\thispagestyle{empty}

\begin{titlepage}

\begin{center}
\vspace*{1in}
	% title
	\centering\textbf{\huge{Topics in advanced modelling techniques\\[1cm]
 	Project finite mixture models
 }} \\[2cm]
	% details
	\Large{
	Master of Statistics and Data Science \\
	Hasselt University\\	
        2024-2025 \\
	}


\vspace*{3cm}
\textbf{\large{Luong Vuong (2365900)}}\\


\vspace*{2cm}


\vspace*{1.5cm}
\textbf{\large{Lecturer:}}\\
Prof. Dr. Geert Verbeke \\

\vspace*{2\baselineskip}
\today

\begin{figure}[b]
    \centering
    \includegraphics[width=8cm]{Plots/UHasselt_logog.png}
    \label{fig:Uhasselt}
\end{figure}

\end{center}

\end{titlepage}

%%% THE WRITTEN PROJECT - MAX. 25 PAGES (everything included)
%%% page numbering starts here.
\newpage \setcounter{page}{1}

\begin{abstract}
\noindent\textbf{Background}:
Hemodialysis patients commonly have low hemoglobin (Hgb) levels, which is adversely related to an increased odds of death. Hemoglobin concentrations in these patients are shown to be associated with factors like Erythropoietin dose, iron status, age, and sex.

\noindent\textbf{Objectives}:
We aimed to investigate the probabilities of being in different Hgb categories over time in hemodialysis patients and evaluate the relationship between these probabilities and Erythropoietin (EPO) dose, iron deficiency status, age, and sex.

\noindent\textbf{Methods}: 


\noindent\textbf{Results}: 


\noindent\textbf{Conclusions}:


 

\end{abstract}
\rule{\textwidth}{0.4pt}

\section{Introduction}\label{introduction}
Renal failure is a common chronic medical condition, with many patients relying on hemodialysis for treatment \cite{KDIGO2024}. These patients are particularly vulnerable to anemia, which can be attributed to various pathophysiological mechanisms. Uremic syndrome, chronic inflammation, erythropoietin (EPO) deficiency, and iron deficiency were identified as key contributors to anemia in this population \cite{Portoles2021}. Hemodialysis patients may present with mild or no anemia, defined as a hemoglobin (Hgb) level higher than 10 g/dL, or moderate to severe anemia, defined as an Hgb level of 10 g/dL or less. This classification can help prompt further investigation or treatment, such as: iron supplement, EPO administration or blood transfusion. In this study, we aimed to investigate the evolution of both the continuous Hgb and the binary Hgb over time in hemodialysis patients and explore how this evolution was influenced by EPO dose, iron deficiency status, age, and sex. 

\section{Data description}\label{datadescription}

The dataset included Hgb concentrations measured longitudinally every month in 3823 patients with renal deficiency receiving hemodialysis. These patients were followed up for a maximum of 6 months. The longitudinal dose of erythropoietin was also collected for these patients, in which the EPO dose for the next month was decided by the Hgb level of the current month. The study design producing the dataset was balanced but the resulting dataset was unbalanced due to missingness. The variables in the dataset were patient id, a month in which measurement was taken (1 to 6), patient age at baseline (year), patient sex, EPO dose to be administered during the following month (IU/kg/week, IU: international Unit), and an indicator of iron deficiency status (1: inadequate iron stores, 0: adequate iron stores) at each month.

\section{Methods}\label{methods}

\subsection{Explore data missingness}  

\subsubsection{Statistical models}
\subsubsection{Data transformation prior to modeling}


\subsection{Linear mixed-effect model (LMM) for Gaussian data}



\subsection{Marginal Models}


\subsubsection*{Proportional Odds Assumption}


\subsubsection*{Empirical Bayes Estimates}


\subsection{Sensitivity analysis}


\section{Results}
\subsection{Explore data missingness}


\subsection{Linear Mixed Models}


\subsubsection*{Comparing results between original and multiply imputed dataset}



\subsection{Generalized Linear Mixed Models}


\subsubsection*{Model incorporating cumulative effect of the time-varying covariate }


\section{Discussion}\label{discussion}

\section{Conclusion}\label{conclusion}


\bibliographystyle{unsrt}
\bibliography{reference}

\newpage
\begin{appendices}
\section*{Appendix 1: Multiple imputation diagnostics}
\label{sec:appendix-mi-diagnostics}

\begin{figure}[H] 
    \centering    \includegraphics[width=0.8\linewidth]{Plots/fig12-mi_convergence_misdose.png}
    \caption{Multiple imputation algorithm has a good convergence on all four missing variables}
    \label{fig:mi-mar-diag-conver}
\end{figure}

\begin{figure}[H] 
    \centering    \includegraphics[width=0.8\linewidth]{Plots/fig13-mi_density_misdose.png}
    \caption{Imputed dose and Hb values are plausible}
    \label{fig:mi-mar-diag-dens}
\end{figure}

\begin{figure}[H] 
    \centering    \includegraphics[width=0.8\linewidth]{Plots/fig14-bwplot_mnar_hb_delta_-4.png}
    \caption{Imputed Hb values are systematically different from the observed ones}
    \label{fig:mi-mnar-diag-box}
\end{figure}

\begin{figure}[H] 
    \centering    \includegraphics[width=0.8\linewidth]{Plots/fig15-densplot_mnar_hb_delta_-4.png}
    \caption{Imputed Hb values are systematically different from the observed ones}
    \label{fig:mi-mnar-diag-dens}
\end{figure}
\end{appendices}
\endpage
\clearpage

\section*{Appendix 2 - Model building stages for generalized linear mixed model}\label{appendix2}


\clearpage
\section*{Appendix 3 - R/SAS code}\label{appendix3}

\lstset{
    language=SAS,
    basicstyle=\ttfamily\small,
    breaklines=true,
    breakatwhitespace=true,
    columns=fullflexible,
    backgroundcolor=\color{white},
    keywordstyle=\color{blue},
    commentstyle=\color{green},
    stringstyle=\color{red},
    showstringspaces=false,
    numbers=left,
    numberstyle=\tiny\color{gray}
}
\textbf{Data transformation}
\begin{lstlisting}


\end{lstlisting} 

\end{document}

